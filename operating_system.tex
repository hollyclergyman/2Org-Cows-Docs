% !TEX root = Documentation.tex

\section{Operating System}
\subsection{Ubuntu}
In Ubuntu less modifications than in CentOS are necessary to allow a stable and safe operation of the server. As the operating system ships another mandatory access controll system
(MAC), configuration of user rights is far easier and faster proceded. During the installation of the operating system, the LAMP stack and the openssh server have been chosen for
automatic installation. After the installation and the required reboot, updates should be checked and installed.\\

\paragraph{Securing the server}
To improve the server security, just logged in users can make use of the server. To prohibit external login attempts using ssh, the server is secured in different manners, of which 
one is the setup of groups, allowed to login using ssh. Therefore, a specified group is created, to which all login users are added, requiring the following commands:
\begin{lstlisting}[language=bash]
 sudo addgroup --system [groupname]
 sudo adduser [username] [groupname]
\end{lstlisting}
Than the /etc/ssh/sshd\_config file is modified, including adding and modifying the following lines in the file:
\begin{lstlisting}
 AllowGroups [groupname]
 PermitRootLogin no
\end{lstlisting}
Finally the ssh server has to be restarted.

\paragraph{Database installation and configuration}
As Ubuntu ships MySQL as its default database, MySQL is used in this installation aswell. The database comes with a basic configuration, which should be checked after installation
to match the conditions, under which the database server is operating.\\
If necessary the configuration file under /etc/mysql/conf.d/server.conf has to be altered. It is important to check, if the networking access to the database server is allowed and 
if, which clients are gained access to the databases. For security and compatibility the setting listen: 127.0.0.1 has been chosen here, which allows the network interface of PHP7
to access the database locally, but hinders external clients from initiating a network connection to the database.
\paragraph{Web server installation and configuration}
Both, apache2 and PHP7 have been installed in the initial installation, together with the database. As the web server should be configured as a virtual host, it is important to 
modify the settings here. There have been two virtual hosts defined, both referring to the same content folder, one host serves as the non-tls demo installation, whereas the other
host is set up with tls support to deliver the results of the requests as encrypted material. The non-tls host is not intended to be used for login, as all user data sent to this host
are in danger of being intercepted. The configuration proceeds in the following steps:
\begin{enumerate}
 \item First the directories for the virtualhost are created, using the following command:\\
 \begin{lstlisting}[language=bash]
  sudo mkdir -p /var/www/path_to_directory/content
 \end{lstlisting}
 \item Next the permissions on the folder are set to the user invoked or a specific user for the deployment of the web server:\\
 \begin{lstlisting}[language=bash]
  sudo chown -R $USER:$USER /var/www/path_to_directory/content
 \end{lstlisting}
 \item As the next step, the configuration files for the virtual host are created. To minimize spelling errors, the files are generated from the default configuration, which is 
 copied into a new file:\\
 \begin{lstlisting}[language=bash]
  sudo cp /etc/apache2/sites-available/000-default.conf /etc/apache2/sites-available/name_virtual_host.conf
 \end{lstlisting}
 \item Now the virtual host has to be configured, with the following settings:\\
 \begin{lstlisting}[language=bash]
  <VirtualHost *:443>
	SSLEngine On
	SSLCertificateFile /etc/ssl/certs/name_of_cert.crt
	SSLCertificateKeyFile /etc/ssl/private/key.key
	SSLCACertificateFile /etc/ssl/certs/certificate.crt
	ServerName www.servername.com
        ServerAlias servername.com
	ServerAdmin admin@e-mail.com
	DirectoryIndex login.php
	ErrorDocument 404 /error.php
	DocumentRoot /var/www/path_to_directory/content
	ErrorLog /var/www/path_to_directory/error.log
	CustomLog /var/www/path_to_directory/access.log combined
	Alias "/admin" /var/www/path_to_directory/content/create-user.php
	Alias "/measurement" /var/www/path_to_directory/content/create-measurement.php
	Alias "/search" /var/www/path_to_directory/content/search.php
	Alias "/results" /var/www/path_to_directory/content/search-results.php
	Alias "/license" /var/www/path_to_directory/content/license.php
	Alias "/upload" /var/www/path_to_directory/content/database-update.php
	Alias "/cow" /var/www/path_to_directory/content/cow.php
	Alias "/logout" /var/www/path_to_directory/content/scripts/logout.php
	Alias "/home" /var/www/path_to_directory/content/home.php
	Alias "/user" /var/www/path_to_directory/content/user-properties.php
  </VirtualHost>
 \end{lstlisting}
  The "Alias" settings maskerade the files, visible to the user with a shorter, easier URL.
  \item In the next step, all files named in the virtual host configuration, must be created or copied to the server. To create the files, the following command is used:\\
  \begin{lstlisting}
   sudo touch /var/www/path_to_directory/error.log
   sudo touch /var/www/path_to_directory/access.log
  \end{lstlisting}
  \item 



\end{enumerate}


\subsection{CentOS 7}
In CentOS 7 several modifications were necessary to provide a stable and safe operation of the database and the interfaces. The following describes the installation of the required
packages to a scratch installation. In case a LAMP installation has been chosen during the installation process of the operating system, some of these steps might be superfluous. 
\paragraph{Database installation and configuration}
First the database had to be installed. 
Therefore \textbf{MariaDB} has been chosen, as it is the default database environment for RHEL-based Linux distributions. For the installation, mariadb-server has to be chosen, all
dependencies are automatically fixed by the yum RPM wrapper. Furthermore PHP7 had to be installed, here it is important
to install the database drivers as well, which are shipped in the package php70w-mysqlnd. Together with PHP7 httpd, as the web server, needs to be installed. For the access of httpd
to the database it is important to execute the following commands:\\
\begin{lstlisting}[language=bash]
 sudo sestatus
 sudo getsebool -a | grep httpd
 sudo setsebool -P httpd_can_network_connect_db 1
\end{lstlisting}
\paragraph{Web server installation and configuration}
As the LAMP stack is used for the project development, \textbf{apache2} is deployed as the default web server. apache2 is installed as a dependency of PHP7 and is shipped in 
the package httpd. To ensure proper operation, the server has to be configured for shipping of multiple websites beside each other, which is ensured by using virtual hosts. 
Therefore the following steps are necessary: 
\begin{enumerate}
 \item create a folder for the virtual host, using the following command:
 \begin{lstlisting}[language=bash]
  sudo mkdir -P /var/www/path_to_directory/content
 \end{lstlisting}
 \item grant the required permissions on the folder to the user:
 \begin{lstlisting}[language=bash]
  sudo chown -R $user:$user /var/www/path_to_directory/content
  sudo chmod -R 755 /var/www/
 \end{lstlisting}
 \item create configuration files for the virtual host:
 \begin{itemize}
  \item Create the folders for the virtual host configuration:
 \begin{lstlisting}[language=bash]
  sudo mkdir /etc/httpd/sites-available
  sudo mkdir /etc/httpd/sites-enabled
 \end{lstlisting}
 \item Add a line at the end of /etc/httpd/conf/httpd.conf:\\
 IncludeOptional sites-enabled/*.conf
 \end{itemize}
\end{enumerate}

