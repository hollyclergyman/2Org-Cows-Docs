% !TEX root = Documentation.tex

\section{Operating System}
\subsection{Ubuntu}
In Ubuntu less modifications than in CentOS are necessary to allow a stable and safe operation of the server. As the operating system ships another mandatory access controll system
(MAC), configuration of user rights is far easier and faster proceded. During the installation of the operating system, the LAMP stack and the openssh server have been chosen for
automatic installation. After the installation and the required reboot, updates should be checked and installed.\\

\subsubsection{Securing the server}
To improve the server security, just logged in users can make use of the server. To prohibit external login attempts using ssh, the server is secured in different manners, of which 
one is the setup of groups, allowed to login using ssh. Therefore, a specified group is created, to which all login users are added, requiring the following commands:
\begin{lstlisting}[language=bash]
 sudo addgroup --system [groupname]
 sudo adduser [username] [groupname]
\end{lstlisting}
Than the /etc/ssh/sshd\_config file is modified, including adding and modifying the following lines in the file:
\begin{lstlisting}
 AllowGroups [groupname]
 PermitRootLogin no
\end{lstlisting}
Finally the ssh server has to be restarted.\\
Another security mechanism is the definition of available services in the /etc/hosts.allow and /etc/hosts.deny files on the server. This allows to block access all other webservices, 
except for the required ones from other IP-addresses. This feature has been implemented in parts.\\
Furthermore, the server makes use of a firewall, to prohibit attacks on not required open ports. Therefore, the ufw interface to iptables has been chosen. As the server is not 
required to listen on other ports than the standard ports for the required services, the general rule is to deny all incoming traffic. At the same time, traffic is generally 
allowed with the following commands:
\begin{lstlisting}[language=bash]
 sudo ufw default deny incoming
 sudo ufw default allow outgoing
\end{lstlisting}
In the next step, all required network services, are excluded from the general denial of the services, using the following commands:
\begin{lstlisting}[language=bash]
 sudo ufw allow ssh
 sudo ufw allow http
 sudo ufw allow https
\end{lstlisting}
After the configuration is finished, the firewall is set active with the following command:
\begin{lstlisting}[language=bash]
 sudo ufw enable
\end{lstlisting}


\subsubsection{Database installation and configuration}
As Ubuntu ships MySQL as its default database, MySQL is used in this installation aswell. The database comes with a basic configuration, which should be checked after installation
to match the conditions, under which the database server is operating.\\
If necessary the configuration file under /etc/mysql/conf.d/server.conf has to be altered. It is important to check, if the networking access to the database server is allowed and 
if, which clients are gained access to the databases. For security and compatibility the setting listen: 127.0.0.1 has been chosen here, which allows the network interface of PHP7
to access the database locally, but hinders external clients from initiating a network connection to the database.

\subsubsection{Web server installation and configuration}
Both, apache2 and PHP7 have been installed in the initial installation, together with the database. As the web server should be configured as a virtual host, it is important to 
modify the settings here. There have been two virtual hosts defined, both referring to the same content folder, one host serves as the non-tls demo installation, whereas the other
host is set up with tls support to deliver the results of the requests as encrypted material. The non-tls host is not intended to be used for login, as all user data sent to this host
are in danger of being intercepted. The configuration proceeds in the following steps:
\begin{enumerate}
 \item First the directories for the virtualhost are created, using the following command:
 \begin{lstlisting}[language=bash]
  sudo mkdir -p /var/www/path_to_directory/content
 \end{lstlisting}
 \item Next the permissions on the folder are set to the user invoked or a specific user for the deployment of the web server:
 \begin{lstlisting}[language=bash]
  sudo chown -R $USER:$USER /var/www/path_to_directory/content
 \end{lstlisting}
 \item As the next step, the configuration files for the virtual host are created. To minimize spelling errors, the files are generated from the default configuration, which is 
 copied into a new file:
 \begin{lstlisting}[language=bash]
  sudo cp /etc/apache2/sites-available/000-default.conf /etc/apache2/sites-available/name_virtual_host.conf
 \end{lstlisting}
 \item Now the virtual host has to be configured, with the following settings:
 \begin{lstlisting}[language=bash]
  <VirtualHost *:443>
	SSLEngine On
	SSLCertificateFile /etc/ssl/certs/name_of_cert.crt
	SSLCertificateKeyFile /etc/ssl/private/key.key
	SSLCACertificateFile /etc/ssl/certs/certificate.crt # can be missing, depending on tls configuration of the server
	ServerName www.servername.com #is missing in case of the server, to allow host resolution to the default virtual host and avoid DNS conflicts
  ServerAlias servername.com #is missing in case of the server, to allow host resolution to the default virtual host and avoid DNS conflicts
	ServerAdmin admin@e-mail.com
	DirectoryIndex login.php
	ErrorDocument 404 /404 #should avoid 404 page not found errors
	DocumentRoot /var/www/path_to_directory/content
	ErrorLog /var/www/path_to_directory/error.log
	CustomLog /var/www/path_to_directory/access.log combined
	Alias "/admin" /var/www/path_to_directory/content/create-user.php
	Alias "/measurement" /var/www/path_to_directory/content/create-measurement.php
	Alias "/search" /var/www/path_to_directory/content/search.php
	Alias "/results" /var/www/path_to_directory/content/search-results.php
	Alias "/license" /var/www/path_to_directory/content/license.php
	Alias "/upload" /var/www/path_to_directory/content/database-update.php
	Alias "/cow" /var/www/path_to_directory/content/cow.php
	Alias "/logout" /var/www/path_to_directory/content/scripts/logout.php
	Alias "/home" /var/www/path_to_directory/content/home.php
	Alias "/user" /var/www/path_to_directory/content/user-properties.php
	Alias "/404" /var/www/path_to_directory/content/error.php # should avoid 404 page not found errors to the error.php page
  </VirtualHost>
 \end{lstlisting}
  The "Alias" settings maskerade the files, visible to the user with a shorter, easier URL. As the php code mostly refers to the shortened Alias URLs, these entries are essential for the server to work. The \textbf{ServerName} and \textbf{ServerAlias} might be missing, to comply with local DNS settings.
  \item In the next step, all files named in the virtual host configuration, must be created or copied to the server. To create the files, the following command is used:
  \begin{lstlisting}
   sudo touch /var/www/path_to_directory/error.log
   sudo touch /var/www/path_to_directory/access.log
  \end{lstlisting}
  \item To ensure logrotate to compress the log files of the virtualhost, the file /etc/logrotate.d/apache2 has to be altered to the following:
  \begin{lstlisting}[language=bash]
   /var/log/apache2/*.log /var/www/path_to_directory/*.log {
        daily
        missingok
        rotate 14
        compress
        delaycompress
        notifempty
        create 640 root adm
        sharedscripts
        postrotate
                if /etc/init.d/apache2 status > /dev/null ; then \
                    /etc/init.d/apache2 reload > /dev/null; \
                fi;
        endscript
        prerotate
                if [ -d /etc/logrotate.d/httpd-prerotate ]; then \
                        run-parts /etc/logrotate.d/httpd-prerotate; \
                fi; \
        endscript
  }
  \end{lstlisting}
  
  \item In the next step, the encryption certificates for the tls service have to be created. The certificates are self-signed certificates, using SHA256 as hashing algorithm and having a validity of 10 years (3650 days). For this, the following command is used:
  \begin{lstlisting}[language=bash]
   openssl req -sha512 -x509 -nodes -days 3650 -newkey rsa:4096 -keyout /etc/ssl/private/[servername].key -out /etc/ssl/certs/[servername].crt
  \end{lstlisting}
  During the creation process, the information for the certificate have to be provided. Also a Diffie-Hellman file has to be created, to improve the security of the tls connection 
  by using the Diffie-Hellman paradigma for PerfectForwardSecrecy of the sessions:
  \begin{lstlisting}[language=bash]
   sudo openssl dhparam -out /etc/ssl/certs/dhparam.pem 4096
  \end{lstlisting}
  \item In the next step, the tls setting of the whole server are modified to meet most recent requirements on the cryptographic strength of the transaction. Therefore, a 
  designated file under the following location /etc/apache2/conf-available/ssl-params.conf is created. The content of the file is the following:
  \begin{lstlisting}[language=bash]
   #modern security configuration
   SSLProtocol             all -SSLv3 -TLSv1 -TLSv1.1
   SSLCipherSuite          ECDHE-ECDSA-AES256-GCM-SHA384:ECDHE-RSA-AES256-GCM-SHA384:ECDHE-ECDSA-CHACHA20-POLY1305:ECDHE-RSA-CHACHA20-POLY1305:ECDHE-ECDSA-AES128-GCM-SHA256:ECDHE-RSA-AES128-GCM-SHA256:ECDHE-ECDSA-AES256-SHA384:ECDHE-RSA-AES256-SHA384:ECDHE-ECDSA-AES128-SHA256:ECDHE-RSA-AES128-SHA256
   SSLHonorCipherOrder     on
   SSLCompression          off
   SSLSessionTickets       off

   # OCSP Stapling, only in httpd 2.3.3 and later
   SSLUseStapling          on
   SSLStaplingResponderTimeout 5
   SSLStaplingReturnResponderErrors off
   SSLStaplingCache        shmcb:/var/run/ocsp(128000)

   SSLOpenSSLConfCmd DHParameters "/etc/ssl/certs/dhparam.pem"
  \end{lstlisting}
  This file defines the encryption standards, which the server accepts for connections. These standards might exclude older clients, such as Internet Explorer on Windows XP.
  \item To enable the configuration, the following commands have to be executed:
  \begin{itemize}
  \item to enable the ssl module of apache2:
  \begin{lstlisting}[language=bash]
   sudo a2enmod ssl
  \end{lstlisting}
  \item to allow redirection from the non-https site to the https site:
  \begin{lstlisting}[language=bash]
   sudo a2enmod headers
  \end{lstlisting}
  \item to enable the https site:
  \begin{lstlisting}[language=bash]
   sudo a2ensite sitename
  \end{lstlisting}
  \item to enable the ssl-config in apache2:
  \begin{lstlisting}[language=bash]
   sudo a2enconf ssl-params
  \end{lstlisting}
  \item and finally the web server has to be restarted:
  \begin{lstlisting}[language=bash]
   sudo systemctl restart apache2
  \end{lstlisting}
  \end{itemize}
  Other files than those mentioned do not need to be altered to allow steady operation of the server. To allow public reaching of the server, the reuqired ports, which is essential 443, have to be cleared by the firewall.
\end{enumerate}

\subsection{CentOS 7}
In CentOS 7 several modifications were necessary to provide a stable and safe operation of the database and the interfaces. The following describes the installation of the required
packages to a scratch installation. In case a LAMP installation has been chosen during the installation process of the operating system, some of these steps might be superfluous. 
\subsubsection{Database installation and configuration}
First the database had to be installed. 
Therefore \textbf{MariaDB} has been chosen, as it is the default database environment for RHEL-based Linux distributions. For the installation, mariadb-server has to be chosen, all
dependencies are automatically fixed by the yum RPM wrapper. Furthermore PHP7 had to be installed, here it is important
to install the database drivers as well, which are shipped in the package php70w-mysqlnd. Together with PHP7 httpd, as the web server, needs to be installed. For the access of httpd
to the database it is important to execute the following commands:\\
\begin{lstlisting}[language=bash]
 sudo sestatus
 sudo getsebool -a | grep httpd
 sudo setsebool -P httpd_can_network_connect_db 1
\end{lstlisting}

\subsubsection{Web server installation and configuration}
As the LAMP stack is used for the project development, \textbf{apache2} is deployed as the default web server. apache2 is installed as a dependency of PHP7 and is shipped in 
the package httpd. To ensure proper operation, the server has to be configured for shipping of multiple websites beside each other, which is ensured by using virtual hosts. 
Therefore the following steps are necessary: 
\begin{enumerate}
 \item create a folder for the virtual host, using the following command:
 \begin{lstlisting}[language=bash]
  sudo mkdir -P /var/www/path_to_directory/content
 \end{lstlisting}
 \item grant the required permissions on the folder to the user:
 \begin{lstlisting}[language=bash]
  sudo chown -R $user:$user /var/www/path_to_directory/content
  sudo chmod -R 755 /var/www/
 \end{lstlisting}
 \item create configuration files for the virtual host:
 \begin{itemize}
  \item Create the folders for the virtual host configuration:
 \begin{lstlisting}[language=bash]
  sudo mkdir /etc/httpd/sites-available
  sudo mkdir /etc/httpd/sites-enabled
 \end{lstlisting}
 \item Add a line at the end of /etc/httpd/conf/httpd.conf:
 \begin{lstlisting}
 IncludeOptional sites-enabled/*.conf
 \end{lstlisting}
 \end{itemize}
\end{enumerate}
